\documentclass{article}
\usepackage[utf8]{inputenc}

\title{On Approximating Non-Transcendental Irrational Numbers in an Optimized Time Frame}
\author{
Anay Aggarwal}
\date{Stoller MS}

\usepackage{natbib}
\usepackage{graphicx}
\usepackage{amsmath}
\usepackage{fancyhdr}
\usepackage{enumitem}
\usepackage{amssymb}
\usepackage{mathptmx}
\usepackage[legalpaper, portrait, margin=1in]{geometry}
\setlength{\parindent}{4em}
\setlength{\parskip}{1em}
\pagestyle{fancy}
\fancyhf{}
\lhead{Running head: Optimizing Approximation}
\renewcommand{\baselinestretch}{2.0}

\begin{document}
\maketitle
\thispagestyle{fancy}
\newpage 
There are multiple ways to approximate irrational numbers such as $\sqrt2$. The first is simply guessing and checking, using brute-force to get the answer (this is very inefficient). The second is Newton's method, which involves using derivatives to get a recursion which approaches an irrational constant as you recurse through again and again (MIT, 2015). The third involves an infinite series of fractions (Rusczyk \& Lehoczky, 2017). Why do we need to approximate these? you may ask. They are very applicable in genetic sequencing (Yau et.al, n.d.) and cryptography. The approximation of algebraic numbers can even be applied to COVID-19! The goal is to find a way to optimize this approximation, be it creating an entirely new way, or comparing current methods.
\par An approximation is useful for multiple non-mathematical fields. One is genetic sequencing. Nucleotides are molecules used for work with genes. The unit vectors representing nucleotides A, G, C and T are $$\hat{A}\left(\frac{1}{2},-\frac{\sqrt3}{2}\right),\hat{G}\left(\frac{\sqrt3}{2},-\frac{1}{2}\right), \hat{C}\left(\frac{\sqrt3}{2},\frac{1}{2}\right),\hat{T}\left(\frac{1}{2},\frac{\sqrt3}{2}\right)$$
(Yau et. al, n.d.). As shown above, they include irrational numbers. Thus, in order to be precise with genetic sequencing, the irrational numbers need to be approximated to a number of digits. In addition, this mostly runs on computers, with large data sets, so time is of the essence. Yau et.al also mentioned that any point $p(x,y)$, where $x$ is the x-projection and $y$ is the y-projection, $2x$ and $2y$ are of the form $m+n\sqrt{3}$ (irrational) where $m,n\in\mathbb{Z}$.
\par Chung, 2003 created a patent using irrational numbers for cryptography, specifically random number generators. Chung, 2003 said, "The improved techniques use irrational numbers over the pseudo-random numbers generated by LFSR and use irrational number generators involve floating-point operations over the conventional integer arithmetic and logic operations.". They also mentioned the applications: "These innovative techniques can be applied to various cryptography applications such as hashes, ciphers, and random number generators. Particularly, the cubic root and inverse cubic root are two suitable functions for use in this invention." (Chung, 2003). This process is essential for high-performance computing. According to (Louie, 2020), simulations for COVID-19 are being run, and the more simulations ran, the faster a vaccine is made. As a result, if one speeds up the time it takes to find the exact value of these numbers, the time it takes to produce a COVID-19 vaccine can be reduced, which is very important in these times.
\par Algebraic numbers are roots of polynomial equations, of the form 
$$f(x)=a_nx^n+a_{n-1}x^{n-1}+a_{n-2}x^{n-2}+...+a_2x^2+a_1x^1+a_0x^0,$$
where $a_i$ is constant $\forall  0\le i\le n$, and $x$ is variant. 
\par Guess and Check is the simplest method to approximate irrational numbers. The name is self-explanatory. "It is essentially guessing an element of a finite set, and testing if it works." (Conway, 2012). For example, let's say we wanted to calculate $\sqrt{2}$. We would start by iterating through a list $\{2,1,0\}$ to find the first perfect square less than or equal to 2. This would be $1$. The general case for $\sqrt{n}$ would be to iterate through $\{n,n-1,...,0\}$. This would give us $1^2<2<2^2$, thus if we wanted to get it to one decimal place we would choose from the set $S_1=\{1.0,1.1,1.2,1.3,1.4,1.5,1.6,1.7,1.8,1.9\}$. The computer would find that $1.4$ is closest, so it will move on to\newline  $S_2=\{1.40,...,1.49\}$. It would then find the closest to $\sqrt{2}$ from that list. This process would continue for as many decimal places as desired.
\par It is not hard to see that the second part (going through the $S_k$ sets) will run in $O(10^n)$ runtime, where $n$ is the number of digits we would want to get it to. This is because $|S_n|=10, \forall n\in\mathbb{Z}^+$. The first part will simply take $O(n+1)$ runtime, however, we can disregard it as it is non-dominant. Thus the total is just $O(10^n)$. 
\par This method is obviously not ideal, since the runtime is exponential. Also, it is not useful for approximating numbers like $\phi,e,$ and $\pi$ (a.k.a Non-Algebraic, Transcendental numbers). Since there is no way to optimize this, it will stand as a baseline to represent the bare minimum.
\par Newton's method for calculating an irrational number uses recursion and brief calculus (UBC, n.d.). Consider $\sqrt{2}\approx 1.4142$. It is a solution to $x^2-2=0$. "Let's choose a linear approximation" (UBC, n.d.). Consider $1$. It is a pretty terrible approximation but we are in the ball park, at least. Call $x_0=1$, and $f(x)=x^2-2$. Thus we get that $f'(x)=2x$ by the power rule for differentiation. We know create a function $g(x)$ defined as $f(x_0)+f'(x_0)(x-x_0)=-1+2*(x-1)=2x-3$. Finding the zero of $g$, we get $\frac32=x_1$, a better approximation for $\sqrt2$. We can rewrite this; "we want $f(x_0)+f'(x_0)(x_1-x_0)=0\rightarrow x_1=x_0-\frac{f(x_0)}{f'(x_0)}$" (UBC, n.d.). We can do the same process to find $x_2$ in terms of $x_1$, and so on. This will give us the general $x_{n+1}=x_n-\frac{f(x_n)}{f'(x_n)}=x_n-\frac{x_n^2-2}{2x_n}$, the essence of Newton's method. As we recurse, we are getting closer and closer to $\sqrt{2}$. We see that $x_2=1.41\overline{6}, x_3\approx 1.4142,$ and so on (UBC, n.d.). In other words, 
$$\lim_{n\to \infty}x_n=\sqrt{2},$$
where $x_0$ is a random constant, and we have the above recursion. 
\par Applying this to a computer program that is optimized for speed is difficult since to find $x_n$, we will need $O(n)$ time. However, the beauty of Newton's method is that it gets many digits, very fast. A formula for the number of digits, $f(n)$, in $n$ steps has not been created yet, but MIT has an approximation. According to (MIT,2015), if $x_n$ converges to $\sqrt{a}$, then the error roughly squares and halves on each iteration.
\par The following is proof of this for $\lim_{n\to\infty}x_n=\sqrt{a}$: 
"\begin{enumerate}
    \item Suppose $x_n>\sqrt{a}$, then it follows $\sqrt{a}<x_{n+1}<x_n$:
        \begin{enumerate}
            \item $x_{n+1}-x_n=\frac{1}{2}(\frac{a}{x_n}-x_n)=\frac{a-x_n^2}{2x_n}<0$.
            \item $x_{n+1}^2-a=\frac{1}{4}(x_n^2+2a+\frac{a^2}{x_n^2})-a=\frac{1}{4}(x_n^2-2a+\frac{a^2}{x_n^2})=\frac{1}{4}(x_n-\frac{a}{x_n})^2=\frac{(x_n^2-a)^2}{4a_n^2}>0$ (regardless of whether $x_n>\sqrt{a}$). 
        \end{enumerate}
    \item A monotonic-decreasing sequence that is bounded below converges (Rudin theorem). $x_1<\sqrt{a}$, the second property above means that $x_n>\sqrt{a}$; then for $n>2$ it is monotonically decreasing and bounded below $\sqrt{a}$.
    \item The limit $x=\lim_{n\to\infty}x_n$ satisfies $x=\frac{1}{2}(x+\frac{a}{x})$, which is easily solved to show that $x^2=a$.$\fbox{}$
\end{enumerate}"(MIT, 2015)
\par Newton also has a method to approximate $\pi$. It involves his binomial expansion:
$$(1+Q)^{\frac{m}{n}}=\sum_{\ell=0}^{\infty}\frac{\prod_{k=0}^{\ell-1}(\frac{m}{n}-k)}{\ell!}Q^{\ell},$$
and brief integration.
\par "We start by labelling $A(0,0), B(1/4,0), C(1/2,0), E(1,0)$ and D on semicircle $AE$ such that $\angle ACD=60^{\circ}$. This semicircle can be described by the equation $y=x^{1/2}(1-x)^{1/2}$. By Newton's Binomial Expansion, this is equal to the following:
$$x^{1/2}(1-x/2-x^2/8-x^3/16-5x^4/128-7x^5/256-...)\approx$$
$$x^{\frac{1}{2}}-\frac{1}{2}x^{\frac{3}{2}}-\frac{1}{8}x^{\frac{5}{2}}-\frac{1}{16}x^{\frac72}-\frac{5}{128}x^{\frac92}-\frac{7}{256}x^{\frac{11}{2}.}$$
We desire $[ABD]$, so we can integrate the following from $0$ to $\frac{1}{4}$:
$$\int_{0}^{1/4}x^{\frac{1}{2}}-\frac{1}{2}x^{\frac{3}{2}}-\frac{1}{8}x^{\frac{5}{2}}-\frac{1}{16}x^{\frac72}-\frac{5}{128}x^{\frac92}-\frac{7}{256}x^{\frac{11}{2}} dx\approx $$
$$0.07677310678\text{ (this part is trivial by the power rule for integration).}$$
Now we have an approximation for the area, so we can use elementary techniques to find the exact area. We have:
$$[ABD]=[ACD]-[\triangle CBD]=\frac{1}{6}\left(\frac{\pi}{2}\right)^2-\frac{1}{2}\cdot \frac14\cdot \frac12\sin(60^{\circ})=\frac{\pi}{24}-\frac{\sqrt{3}}{32}.$$
Now, we can equate the two: $$\frac{\pi}{24}-\frac{\sqrt3}{32}\approx 0.0767731068.$$ Solving for $\pi$ (using Newton's method to calculate $\sqrt{3}$), we get $\pi\approx \textbf{3.1415926}68$ (the correct numbers are in boldface)"
(Norse, 2020)
\par Minimal research has been done in continued fractions for the approximation of irrational numbers, however, irrational numbers can be represented by continued fractions. A continued fraction is of the form:
$$a+\frac{b}{c+\frac{d}{e+\frac{f}{g+...}}}.$$
In this chapter, we will mostly be working with proper continued fractions, of the form:
$$a+\frac{1}{b+\frac{1}{c+\frac{1}{d+...}}}.$$
These are very useful for irrational numbers. We represent it as $[a; b,c,d,...]$. "All irrational numbers have continued fractions that are infinite, and all rational numbers have continued fractions that are finite." (Rusczyk et.al, 2017). 
\par The continued fraction 
$$12+\frac{1}{2+\frac{1}{1+\frac{1}{2+\frac{1}{1+\frac{1}{1+\frac{1}{17}}}}}},$$
can be "written in an abbreviated notation: $12\frac{1}{2+}\frac{1}{1+}\frac{1}{2+}\frac{1}{1+}\frac{1}{1+}\frac{1}{17}$" (Conway, 2012, pg. 178). Here, $12,2,1,2,1,1,17,2$ are $\textbf{partial quotients}$.
\par For example, we see that the irrational $\sqrt2$ is represented by the following:
$$1+\frac{1}{2+\frac{1}{2+...}}.$$
This is useful because as we evaluate more and more terms, we get closer and closer to the actual value of $\sqrt{2}$, thus giving us an approximation. To show this, we can let the expression equal to a variable, say $x$. Now we must make the key observation and note that 
$$x=1+\frac{1}{1+x}\rightarrow x^2+x=x+2\rightarrow x^2=2,$$
which implies the result. Let's say we wanted to reverse the process, find the continued fraction for $\sqrt3$. Note that $\sqrt{3}$ is the solution to $x^2=3\rightarrow x^2+x=x+3\rightarrow x(x+1)=2+(x+1)$. Dividing both sides by $x+1$ gives $x=1+\frac{2}{x+1}$ (the purpose of this was to get x in the denominator, with a constant in front of the fraction). Now it is not hard to see that $x=[1;1,2,1,2,1,2,...]$. "Lagrange [a mathematician] proved that continued fractions are periodic [repeating] just for algebraic numbers [solutions of a polynomial equation] of degree 2" (Conway, 2012, pg. 186). The square root of two has period 1, and the square root of 3 has period 2, as shown above.
\par Continued fractions are also useful for representing numbers such as $e$ (but not $\pi$), unlike some other methods. The continued fraction representation for Napier's $e$ is the following:
$$2+\frac{1}{1+}\frac{1}{2+}\frac{1}{1+}\frac{1}{1+}\frac{1}{4+}\frac{1}{1+}\frac{1}{1+}\frac{1}{6+}\frac{1}{1+}\frac{1}{1+}\frac{1}{8+},$$
where after the third partial quotient, it repeats as $[1,1,2n]$.
\par The problem with this is that it requires intuition, which a computer does not have. With Machine Learning, however, this can be done. Also, (Chung, 2003) says that $\sqrt[3]{2}$ doesn't show any repetitive pattern in its continued fraction representation, along with other algebraic numbers of degree 3 or higher; continued fractions are most useful for square-roots. 
\par Approximating irrational numbers, specifically algebraic numbers, is very important for many things. These include genetic sequencing (Yau et.al, n.d.), cryptography (Chung, 2003), high-performance computing, and even COVID-19 (Louie, 2020). For genetic sequencing, this is because the vectors that genetics experts use to model their data involves irrationals such as $\sqrt{3}$ (Yau et.al, n.d.). For cryptography, irrationals are useful for producing efficient random number generators (Chung, 2003). This can be used for high-performance computing, which can be applied to the production of a potential vaccine for COVID-19. There are multiple methods to approximate the numbers. The first, and most inefficient, is to simply guess and check numbers until you get a result that is correct to $n$ digits. This runs in $O(10^n)$ runtime. The second method was created by Isaac Newton. The essence of it is creating a recursive sequence that tends to an irrational number using functional manipulation and some calculus. This can also be used to approximate $\pi$. The runtime for this method depends on how efficient the method is implemented. The convergence rate is variant for this method. The third is using infinite continued fraction expansions and evaluating just a few terms for an approximation. However, this requires intuition, which will need artificial intelligence to run. It also is not useful for many algebraic numbers of degree $\ge 3$, as they are not periodic $\pmod{k}$ for an integral $k$. There are interesting applications of this to transcendental numbers, $e$, the base of the natural logarithm, in specific. Again, the runtime for this method depends on the implementation in code. For my project, I will compare methods for approximating these numbers, convert them to code, and test the runtimes. I may come up with a method of my own to approximate irrational numbers.


\newpage 
\section*{References}



\begin{enumerate}[leftmargin=!,labelindent=5pt,itemindent=-15pt]
    \item Rusczyk, R., \& Lehoczky, S. (2017). The Art of Problem Solving: Volume 2 and Beyond. Alpine, CA: AoPS.

    \item Conway, J. H., \& Guy, R. K. (2012). The Book of Numbers. Place of publication not identified: Springer.

    \item Newton's Method. (n.d.). Retrieved November 08, 2020, from \newline https://www.ugrad.math.ubc.ca/coursedoc/math100/notes/approx/newton.html

    \item MIT (2015, February 4). Newton's Method for Approximation. Retrieved November 07, 2020, from \newline https://math.mit.edu/~stevenj/18.335/newton-sqrt.pdf
    
    \item Norse, M. (September 5th, 2020). Intermediate Algebra Lecture.
    
    \item Yau, S., Ho, Y., Jin, N., Lu, C., Niknejad, A., \& Wang, J. (n.d.). DNA sequence representation without degeneracy. Retrieved November 09, 2020, from https://academic.oup.com/nar/article/31/12/3078/1395335
    
    \item Chung, S. (2003, January 23). US20030016823A1 - Method and apparatus of using irrational numbers in random number generators for cryptography. Retrieved November 09, 2020, from \newline https://patents.google.com/patent/US20030016823A1/en
    
    \item Louie, S. B. (2020, October). In the pursuit of COVID-19 treatments, high-performance computing steps up. Retrieved November 09, 2020, from https://www.oracle.com/news/connect/gridmarkets-speeds-drug-discovery-with-oracle-cloud-hpc.html?source=\%3Aso\%3Ach\%3Aor\%3Aawr\%3A\%3A\%3
\end{enumerate}

\section*{Acknowledgements}
Thanks to my parents, and my classmates, specifically Sohan Govindaraju and Katie Jin, for proofreading this paper.
\\
A big thanks to Chirag Samantaroy for correcting many grammatical errors and helping with mathematical formatting.
\end{document}

